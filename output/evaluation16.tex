\documentclass[12pt,a4paper]{article}
\setlength{\parindent}{0em}
\usepackage{array}
\newcolumntype{P}[1]{>{\centering\arraybackslash}b{#1}}


\begin{document}
\thispagestyle{empty}
Ecole de la transition\hfill Juin 2022
\bigskip
\begin{center}  
 \begin{Large}  
 \textbf{Évaluation en mathématiques}
\end{Large}  
\end{center}
\medskip

\begin{center}  
    \begin{large}  
       Jonathan Montalvo Fernandez (6.C)
   \end{large}  
   \end{center}
   \medskip
   \vspace{12pt}
\bgroup
\def\arraystretch{1.25}%
\begin{tabular}{ |l |P{2.5cm}| P{2.5cm}| }
    \cline{2-3}
    \multicolumn{1}{l|}{\textbf{Calcul}}&{\small \textit{Sans consigne en français}} & {\small \textit{Avec consignes en français}} \\
    \hline
    
Nombres entiers et décimaux&EA&\\
\hline
Nombres positifs et négatifs&A&M\\
\hline
Fractions&NA&\\
\hline
Puissances et racines&&\\
\hline
Proportionnalités&&\\
\hline
\multicolumn{3}{l}{\textbf{Espace}}\\
\hline
Transformations géométriques&A&M\\
\hline
Polygones&&\\
\hline
Cercles et angles&&\\
\hline
Solides&&\\
\hline\multicolumn{3}{l}{\textbf{Algèbre}}\\
\hline
Calcul littéral&A&NA\\
\hline
Équations&&\\
\hline
Fonctions&&\\
\hline\multicolumn{3}{l}{\textbf{Grandeurs et mesures}}\\
\hline
Grandeurs, mesures et conversions&&\\
\hline
Théorèmes de Pythagore et Thalès&EA&\\
\hline
Surfaces et Volumes&&\\
\hline
Trigonométrie&&\\
\hline
   
  \end{tabular}
  \egroup
  \\
 \begin{center}
  L'enseignant : Mathieu Cossutta\\
  \bigskip
  (NA : non acquis, EA : en cours d'acquisition, A : acquis, M : maîtrisé)
\end{center}

  \end{document}
