\documentclass[12pt,a4paper]{article}
\setlength{\parindent}{0em}
\usepackage{array}
\newcolumntype{P}[1]{>{\centering\arraybackslash}b{#1}}


\begin{document}
\thispagestyle{empty}
Ecole de la transition\hfill Juin 2022
\bigskip
\begin{center}  
 \begin{Large}  
 \textbf{Évaluation en mathématiques}
\end{Large}  
\end{center}
\medskip

\begin{center}  
    \begin{large}  
       $ELEVES$
   \end{large}  
   \end{center}
   \medskip
   \vspace{12pt}
\bgroup
\def\arraystretch{1.25}%
\begin{tabular}{ |r |l |P{2.5cm}| P{2.5cm}| }
    \cline{3-4}
    \multicolumn{2}{l|}{\textbf{Calcul}}&{\small \textit{Sans consigne en français}} & {\small \textit{Avec consignes en français}} \\
    \hline
1&Nombres entiers et décimaux&NA&\\\hline
2&Nombres positifs et négatifs&M&EA\\\hline
3&Fractions&M&A\\\hline
4&Puissances et racines&&NA\\\hline
5&Proportionnalités&NA&\\
    \hline
    \multicolumn{4}{l}{\textbf{Espace}}\\
    \hline
    6 & Notions de base &M&M\\
    \hline
    7 & Symétries et translations &A& NA\\
    \hline
    8 & Triangles et quadrilatères &M&A\\
    \hline
    9 & Déductions d'angles &M&A\\
    \hline
    10 & Solides et développement &A&NA\\
    \hline
    \multicolumn{4}{l}{\textbf{Algèbre}}\\
    \hline
    11 & Calcul littéral &A&A\\
    \hline
    12 & Équations du 1er degré&M&A\\
    \hline
    13 & Fonctions et diagrammes&M&M\\
    \hline
    \multicolumn{4}{l}{\textbf{Grandeurs et mesures}}\\
    \hline
    14 & Grandeurs, mesures et conversions&NA&M\\
    \hline
    15&Théorèmes de Pythagore&M&NA\\
    \hline
    16 & Périmètres et surfaces&A&A\\
    \hline
    17 & Volumes &A&A\\
    \hline
  \end{tabular}
  \egroup
  \\
 \begin{center}
  L'enseignant : $PROFESSEUR$\\
  \bigskip
  (NA : non acquis, EA : en cours d'acquisition, A : acquis, M : maîtrisé)
\end{center}

  \end{document}
