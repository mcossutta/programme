\documentclass[11pt,a4paper]{article}
\usepackage{amsfonts,amsmath,amssymb,mathrsfs}
\usepackage{graphicx}
\usepackage[T1]{fontenc}
\usepackage[utf8]{inputenc}
\usepackage[useregional]{datetime2}
\usepackage{answers}
\usepackage[thmmarks,amsmath]{ntheorem}
\usepackage{fancyhdr}
\usepackage{fancyhdr}
\usepackage{xpatch}
\usepackage{fullpage}
\begin{document}

\begin{titlepage}
    \setlength{\topskip}{0mm}
Nom: \hfill \the\year-C
\vspace{0.5cm}
\hrule

    \centering
    \vspace{0.1\textheight}
	{\bfseries\scshape\Huge 2 Nombres positifs et négatifs\par}
    \vspace{1.5cm}

    {\huge }
	\vspace{1.5cm}
    
    {\large Je suis capable de :}
    \vspace{1.5cm}
        %\setlength{\itemindent}{3cm}
    \begin{itemize}
\item Placer un nombre relatif sur la droite numérique et repérer un point dans le plan.
\item Ordonner une suite de nombres relatifs.
\item Calculer les 4 opérations sur les nombres relatifs en passant par la valeur absolue.
\item Utiliser la priorité des opérations avec des nombres relatifs.
\item Résoudre des problèmes avec des nombres relatifs.\end{itemize}

\vfill
\hrule
\vspace{0.5cm}
%\DTMsetdatestyle{ddmmyyyy}
%\today{}
Arithmétique\hfill MCT-EdA

\end{titlepage}
\end{document}