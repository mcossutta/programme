\documentclass[11pt,a4paper]{article}
\usepackage{amsfonts,amsmath,amssymb,mathrsfs}
\usepackage{graphicx}
\usepackage[T1]{fontenc}
\usepackage[utf8]{inputenc}
\usepackage[useregional]{datetime2}
\usepackage{answers}
\usepackage[thmmarks,amsmath]{ntheorem}
\usepackage{fancyhdr}
\usepackage{fancyhdr}
\usepackage{xpatch}
\usepackage{fullpage}
\begin{document}

\begin{titlepage}
    \setlength{\topskip}{0mm}
Nom: \hfill \the\year-C
\vspace{0.5cm}
\hrule

    \centering
    \vspace{0.1\textheight}
	{\bfseries\scshape\Huge 4 Puissances et racines\par}
    \vspace{1.5cm}

    {\huge Notation scientifique}
	\vspace{1.5cm}
    
    {\large Je suis capable de :}
    \vspace{1.5cm}
        %\setlength{\itemindent}{3cm}
    \begin{itemize}
\item Passer de la notation décimal à la notation scientifique et inversement
\item Utiliser les propriétés des puissances pour faire des calculs avec des nombres en notation scientifique.
\item Résoudre un problème de physique utilisant des nombres en notation scientifique.\end{itemize}

\vfill
\hrule
\vspace{0.5cm}
%\DTMsetdatestyle{ddmmyyyy}
%\today{}
Arithmétique\hfill MCT-EdA

\end{titlepage}
\end{document}